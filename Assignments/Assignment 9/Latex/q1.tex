\section*{سوال ۱}

الگوریتم زمان‌بندی
\lr{Rate Monotonic}
را برای هر مجموعه وظیفه
\lr{periodic}
دل‌خواهی پیاده‌سازی کنید.

\section*{جواب سوال ۱}

الگوریتم زمانبندی 
\lr{Rate Monotonic}
یکی از الگوریتم‌های زمانبندی استاتیک برای سیستم‌های واقع‌زمانی است که در آن وظایف بر اساس دوره تناوب
\lr{(period)}
خود مرتب می‌شوند. وظایف با دوره کوتاه‌تر اولویت بالاتری دارند. هر وظیفه دارای دوره تناوبی است که در آن باید حداقل یک بار اجرا شود. نکته کلیدی در این الگوریتم این است که اولویت وظایف ثابت است و تغییر نمی‌کند.


در پیاده‌سازی ارائه شده، وظایف ابتدا بر اساس دوره تناوبشان مرتب می‌شوند. سپس، سیستم در یک حلقه بی‌پایان وظایف را بررسی می‌کند و وظیفه‌ای را اجرا می‌کند که نوبت آن رسیده است. به‌روزرسانی زمان‌بندی وظایف نیز در هر دور مورد توجه قرار می‌گیرد تا اطمینان حاصل شود که هر وظیفه در دوره خود به طور کامل اجرا می‌شود.


برای نمایش عملکرد الگوریتم زمان‌بندی 
\lr{Rate Monotonic}
در اینجا یک نمونه سناریو با سه وظیفه ذکر می‌شود:

\begin{itemize}
	\item وظیفه A : دوره تناوب = 5 واحد زمان، زمان اجرا = 2 واحد زمان
	\item وظیفه B : دوره تناوب = 10 واحد زمان، زمان اجرا = 1 واحد زمان
	\item وظیفه C : دوره تناوب = 15 واحد زمان، زمان اجرا = 3 واحد زمان
\end{itemize}

\subsection*{خروجی مورد انتظار}

بر اساس الگوریتم زمان‌بندی 
\lr{Rate Monotonic}
برنامه‌ریزی زمانی ممکن است به شکل زیر باشد:
\begin{itemize}
	\item زمان 0-1: وظیفه A
	\item زمان 2-3: وظیفه A
	\item زمان 4: وظیفه B
	\item زمان 5-6: وظیفه A
	\item زمان 7-8: وظیفه A
	\item زمان 9: وظیفه B
	\item ...
\end{itemize}