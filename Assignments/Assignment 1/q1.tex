\section*{سوال ۱}

قسمت‌های مکانیکی و الکترونیکی یک سامانه با چه توزیع احتمالاتی مدل می‌شوند و تفاوت آن‌ها در چیست؟

\section*{جواب سوال ۱}

ابتدا به تعریف این دو می‌پردازیم، سپس به بیان توزیع‌ها احتمالاتی که با آن‌ها مدل می‌شوند، مقایسه‌ی تفاوت‌های آن‌ها و در نهایت دلیل استفاده از هر کدام از این توزیع‌ها می‌پردازیم.

\subsection*{تعاریف}

قسمت‌های مکانیکی به اجزای فیزیکی یک سامانه اطلاق می‌شود که وظیفه‌ی تحمل، انتقال، یا تولید نیروها، حرکت‌ها، و انرژی‌های مکانیکی را دارند و معمولا از موادی با خواص مکانیکی معین ساخته شده‌اند. این قسمت‌ها به طور مستقیم در تعامل با محیط فیزیکی و دیگر اجزا سامانه قرار دارند و عملکرد آنها معمولا با قوانین مکانیک کلاسیک قابل توصیف است.

قسمت‌های الکترونیکی به اجزا و مولفه‌های فیزیکی یک سامانه اطلاق می‌شود که با استفاده از ویژگی‌های الکترونیکی مواد، سیگنال‌ها و انرژی‌های الکتریکی را پردازش، کنترل، انتقال یا تبدیل می‌کنند. این اجزا به طور مستقیم با سیگنال‌های الکتریکی وارد تعامل می‌شوند و عملکرد آن‌ها اغلب بر اساس قوانین فیزیک کوانتومی و نیمه‌هادی‌ها قابل توصیف است.

\subsection*{توزیع‌های احتمالاتی مورد استفاده برای مدل‌سازی}

حالا به بیان توزیع‌های احتمالاتی که با آن‌ها مدل می‌شوند، می‌پردازیم.

قسمت‌های مکانیکی معمولاً به تحمل بار، خزند، خستگی، و سایر مسائل مکانیکی پاسخ می‌دهند.
توزیع‌های مرتبط با عمر خستگی، زمان به خرابی، و توزیع‌های مرتبط با خزند و خستگی معمولاً به کمک توزیع‌های
\lr{Weibull}
، لوگ-نرمال یا توزیع‌های نمایی مدل می‌شوند.

قسمت‌های الکترونیکی ممکن است به خطاهایی مواجه شوند که ناشی از نویز، تغییرات در ولتاژ، یا افت‌هایی در توان هستند.
توزیع‌های گوسی یا نرمال معمولاً برای مدل‌سازی نویز‌ها و خطاها در سیستم‌های الکترونیکی استفاده می‌شود.
برای مدل‌سازی زمان تا خرابی یا عمر مفید قطعات الکترونیکی نیز می‌توان از توزیع‌های
\lr{Weibull}
یا نمایی استفاده کرد.

\subsection*{تفاوت‌های این توزیع‌ها}

\subsubsection*{مکانیکی}

پاسخ‌های فیزیکی به بارهای خارجی

خزند، خستگی، تنش‌ها و کرنش‌ها

مواد، جوانه‌ها، ترک‌ها و سایر عوامل مکانیکی

\subsubsection*{الکترونیکی}

پاسخ‌های الکتریکی به سیگنال‌ها و تغییرات ولتاژ

نویز، تغییرات فرکانس، و سایر مسائل مرتبط با جریان و ولتاژ

مدارات، المان‌ها، و تجهیزات الکترونیکی

در نهایت، توزیع احتمالاتی مناسب برای مدل‌سازی واحدهای مکانیکی یا الکترونیکی بستگی به داده‌ها، تجربیات و مسائل خاصی دارد که می‌خواهیم مورد بررسی قرار دهیم.

قسمت‌های مکانیکی معمولاً به تحمل بار، خزند، خستگی، و سایر مسائل مکانیکی پاسخ می‌دهند.
توزیع‌های مرتبط با عمر خستگی، زمان به خرابی، و توزیع‌های مرتبط با خزند و خستگی معمولاً به کمک توزیع‌های 
\lr{Weibull}
، لوگ-نرمال یا توزیع‌های نمایی مدل می‌شوند.

درباره‌ی سطح کارکرد باید گفت که، قطعات الکترونیکی در سطوح میکرو و نانو فعالیت دارند و از قوانین فیزیک کوانتومی پیروی می‌کنند، در حالی که قطعات الکتریکی در سطح ماکرو عمل می‌کنند و از قوانین الکترومغناطیس پیروی می‌کنند.

همچنین درباره‌ی نوع خطاها و خرابی‌ها هم باید گفت که قطعات الکترونیکی معمولاً به خطاهایی ناشی از نویز‌ها، تغییرات فرکانس یا تغییرات ولتاژ حساس هستند. در مقابل، قطعات الکتریکی به خطاهایی ناشی از تغییرات

\subsection*{دلایل استفاده از هر کدام از این توزیع‌ها}

\subsubsection*{توزیع
\lr{Weibull}
}


این توزیع به خوبی می‌تواند رفتار مواد در شرایط مختلف فشار، دما یا سایر شرایط فیزیکی را توصیف کند.

استفاده‌های معمول در مهندسی مکانیک شامل تجزیه عمر خستگی، زمان تا خرابی و خزند است.

در مهندسی الکترونیک نیز برای توصیف عمر مفید قطعات در برابر عوامل مخرب مانند حرارت یا تغییرات ولتاژ مورد استفاده قرار می‌گیرد.

\subsubsection*{لوگ-نرمال}

توزیع لوگ-نرمال برای متغیرهایی که مقادیر آنها همیشه مثبت هستند، مناسب است.

این توزیع می‌تواند انحرافات از نرمالیت را توصیف کند، که در بسیاری از مسائل مهندسی مکانیک مانند خزند یا خستگی مورد استفاده قرار می‌گیرد.

\subsubsection*{توزیع نمایی}

توزیع نمایی برای مدل‌سازی زمان بین دو رویداد بدون حافظه مانند خرابی‌ها یا وقوع افت‌های برق مناسب است.

این توزیع می‌تواند نشان‌دهنده‌ی توزیع زمان تا اولین خرابی یا وقوع رویداد باشد.

\subsubsection*{توزیع گوسی یا نرمال}


برای مدل‌سازی نویز‌ها و خطاها در سیستم‌های الکترونیکی به خوبی مناسب است، زیرا بسیاری از پدیده‌های طبیعی به این توزیع نزدیک هستند.

این توزیع اغلب در مدل‌سازی متغیرهایی که نتیجه‌ی ترکیب چندین عامل مستقل هستند، مورد استفاده قرار می‌گیرد.
