در ابتدا به بررسی پارادایم دیفرانسیلی این سوال می‌پردازیم. در اینجا با توجه به تابع تبدیل داریم:

\begin{flalign*}
    \theta(s) &= \frac{Ap}{s(s + p)} \\
    \text{PID}(s) &= \frac{1}{s + 4.54} \\
    \text{Filter}(s) &= \frac{1}{s + 1} 
\end{flalign*}

توابع تبدیل موجود در سیستم، نقش کلیدی در پاسخ دینامیکی سیستم به ورودی‌های مختلف دارند. به طور مثال، 
\lr{PID(s)}
با پارامتر تنظیم شده برای دستیابی به یک پاسخ سریع و بدون اُسیلاسیون در حالت تعادل است.


بنابراین برای طراحی دیاگرام آن نیاز به یک فیلتر،
Grain ،
تابع تبدیل و 
PID
کنترلر داریم. در اینجا سینگال ورودی تابع پله بوده و با استفاده از بلوک
Mux
در
Scope
توانسته‌ایم که سیگنال خروجی را براساس ورودی رسم کنیم. شکل دیاگرام طراحی شده برای این قسمت در پایین قابل مشاهده است.

\vspace{-10pt}

\begin{figure}[H]
    \centering
    \includegraphics[width = 0.8\textwidth]{commons/image1.png}
    \caption{دیاگرام پارادایم دیفرانسیلی}
\end{figure}

همچنین خروجی این پارادایم به این صورت قابل مشاهده است:

\begin{figure}[H]
    \centering
    \includegraphics[width = 0.6\textwidth]{commons/image2.png}
    \caption{خروجی پارادایم دیفرانسیلی}
\end{figure}

همانطور که در خروجی این قسمت می‌بینیم، سیگنال خروجی به صورت پیوسته در حال تغییر است و پس از دریافت سیگنال پله به عنوان ورودی، پس از مدتی به حالت تعادل خود رسیده است.

در این بخش به طراحی پارادایم
\lr{State Chart}
می‌پردازیم. کاری که باید در اینجا انجام دهیم این است که براساس دما حالت‌های مختلف را ایجاد کنیم و رفتاری گسسته‌وار در مقابل تغییرات دما داشته باشیم.

در اینجا ما دو کامپوننت
Cooler
و
Heater
داریم که براساس شریط 
Temperature
در حال جابه‌جایی بین استیت‌های مختلف هستند. سینگال ورودی در این سیستم از نوع تابع
\lr{Sinc wave}
بوده و در خروجی شاهد حالات
Cooler
و
Heater
براساس ورودی هستیم.

شکل
\lr{State Flow}
طراحی شده برای این منظور را به این صورت می‌توانید مشاهده کنید:

\begin{figure}[H]
    \centering
    \includegraphics[width = 0.8\textwidth]{commons/image3.png}
    \caption{طراحی استیت‌های سیستم کنترل دما}
\end{figure}

همچنین دیاگرام کلی به این صورت خواهد بود:

\begin{figure}[H]
    \centering
    \includegraphics[width = 0.8\textwidth]{commons/image4.png}
    \caption{دیاگرام کلی مدل}
\end{figure}

\newpage

در نهایت برای خروجی‌های
Cooler
و
Heater
خواهیم داشت:

\begin{figure}[H]
    \centering
    \includegraphics[width = \textwidth]{commons/image5.png}
    \caption{خروجی مدل استیت چارت}
\end{figure}

همانگونه که مشاهده می‌کنید در اینجا براساس ورودی دما یا در حالتی هستیم که 
Heater
روشن بوده یا
Cooler
و یا اینکه هر دو خاموش هستند.

تفاوت اصلی این دو پارادایم در نوع پاسخ سیستم به تغییرات است. در پارادایم براساس معادلات دیفرانسیلی ما شاهد این هستیم که سیستم پاسخی پیوسته و همواره در حال تغییر به ورودی ما می‌دهد. در صورتی که در پارادایم استیت چارت خروجی‌های به صورت صفر و یک بوده و صرفا مشخص می‌کند که براساس ورودی داده شده باید در چه حالت‌های از پیش تعیین شده‌ای قرار بگیریم.

به این ترتیب با توجه به نوع سیستمی که در نظر داریم پیاده کنیم و نوع سیگنال ورودی می‌بایست یکی از این دو نوع پارادایم را برای مدل‌سازی سیستم‌مان در نظر بگیریم و هر کدام از پارادایم‌ها کاربرد به خصوص خودشان را خواهند داشت.

در نمودار پارادایم دیفرانسیلی، هر بلوک نشان‌دهنده یک بخش خاص از سیستم کنترل است. به طور مثال، بلوک
\lr{Mux}
وظیفه جمع‌آوری سیگنال‌های ورودی و ارسال آن‌ها به بلوک 
\lr{Scope}
را دارد.

مقاومت سیستم در برابر اختلالات و تغییرات محیطی از جنبه‌های کلیدی در طراحی کنترلرهای مؤثر است. این بخش به بررسی پایداری سیستم پس از اعمال اختلالات خارجی و تعیین حدودی که سیستم می‌تواند بدون از دست دادن کنترل به فعالیت ادامه دهد، می‌پردازد.

مدل‌سازی سیستم‌های کنترل می‌تواند از روش‌های خطی و غیرخطی استفاده کند. در حالی که مدل‌های خطی به دلیل سادگی تحلیل و طراحی محبوب هستند، مدل‌های غیرخطی امکان دارند پدیده‌های پیچیده‌تری مانند اشباع و هیسترزیس را بهتر توصیف کنند. این بخش به مقایسه این دو روش و بررسی موقعیت‌هایی که استفاده از هر یک توصیه می‌شود می‌پردازد.